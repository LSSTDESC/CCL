
\begin{table*}[t]
  \centering
  \begin{tabular}{ l c | c c c c c c c c c c }
    \hline
    \multicolumn{12}{|c|}{Cosmological models with massive neutrinos} \\
    \hline
    \hline
    Acronym & Model & $\Omega_m$ & $\Omega_b$ & $\Omega_\Lambda$ & $h_0$ & $\sigma_8$ & $n_s$ & $w_0$ & $w_a$ & $N_{\rm eff}$ & $m_\nu$ (eV) \\
    \hline
    CCL7 & flat $\Lambda$CDM, $m_\nu$ & 0.3 & 0.05 & 0.7 & 0.7 & 0.8 & 0.96 & -1 & 0 & 3.013 & \{0.04, 0, 0\} \\
    CCL8 & $w$CDM, $m_\nu$ & 0.3 & 0.05 & 0.7 & 0.7 & 0.8 & 0.96 & -0.9 & 0 & 3.026 & \{0.05, 0.01, 0\} \\
    CCL9 & $w$CDM, $m_\nu$ & 0.3 & 0.05 & 0.7 & 0.7 & 0.8 & 0.96 & -0.9 & 0.1 & 3.040 & \{0.03, 0.02, 0.04\} \\
    CCL10 & open $w$CDM, $m_\nu$ & 0.3 & 0.05 & 0.65 & 0.7 & 0.8 & 0.96 & -0.9 & 0.1 & 3.013 & \{0.05, 0, 0\}  \\
    CCL11 & closed $w$CDM, $m_\nu$ & 0.3 & 0.05 & 0.75 & 0.7 & 0.8 & 0.96 & -0.9 & 0.1 & 3.026 &\{0.03, 0.02, 0\} \\
    \hline
  \end{tabular}
  \caption{Cosmological models with massive neutrinos used in testing \ccl against independently produced benchmarks. We calculate $N_{\rm eff}$ according to Eq.~(\ref{Nnurel}), based on the number of massless and massive neutrino species.
}
  \label{tab:cosmologies_nu}
\end{table*}

% 
% ======================================================================
\RequirePackage{docswitch}
% \flag is set by the user, through the makefile:
%    make note
%    make apj
% etc.
\setjournal{\flag}

\documentclass[\docopts]{\docclass}

% You could also define the document class directly
%\documentclass[]{emulateapj}

% Custom commands from LSST DESC, see texmf/styles/lsstdesc_macros.sty
\usepackage{lsstdesc_macros}

\usepackage{graphicx}
\graphicspath{{./}{./figures/}}
\bibliographystyle{apj}

% Add your own macros here:



% 
% ======================================================================

\begin{document}

\title{ {{ cookiecutter.title }} }

\maketitlepre

\begin{abstract}

{{ cookiecutter.description }}

\end{abstract}

% Keywords are ignored in the LSST DESC Note style:
\dockeys{latex: templates, papers: awesome}

\maketitlepost

% ----------------------------------------------------------------------
% 

\section{Introduction}
\label{sec:intro}

This is a paper and note template for the LSST DESC \citep{Overview,ScienceBook,WhitePaper}.
You can delete all this tutorial text whenever you like.

You can easily switch between various \LaTeX\xspace styles for internal notes and peer reviewed journals.
Documents can be compiled using the provided \code{Makefile}.
The command \code{make} with no arguments compiles \code{main.tex} using the  \code{lsstdescnote.cls} style.
If you want to upgrade your Note into a journal article, just choose a journal name, between \code{make apj} (ApJ preprint format), \code{make apjl} (which uses the \code{emulateapj} style), \code{make prd}, \code{make prl}, and \code{make mnras}.


% ----------------------------------------------------------------------

\section{Commands}
\label{sec:commands}

There are a number of useful \LaTeX\xspace commands predefined in \code{macros.tex}.
Notice that the section labels are prefixed with \code{sec:} to allow the use of the \verb=\secref= command to reference a section (\ie, \secref{intro}).
Figures can be referenced with the \verb=\figref= command, which assumes that the figure label is prefixed with \code{fig:}.
In \figref{example} we show an example figure.
You'll notice that the actual figure file is found in the \code{figures} directory.
However, because we have specified this directory in our \verb=\graphicspath= we do not need to explicitly specify the path to the image.

The \code{macros.tex} package also contains some conventional scientific units like \angstrom, \GeV, \Msun, etc. and some editorial tools for highlighting \FIXME{issues}, \CHECK{text to be checked}, \COMMENT{comments}, and \NEW{new additions}.


% ----------------------------------------------------------------------

\section{Methods}
\label{sec:methods}

Similar to the figure before, here we have included a table of data from \code{tables/table.tex}.
Notice that again we are able to reference \tabref{example} with the \verb=\tabref= command using the \code{tab:} prefix.
Also notice that we haven't needed to specify the full path to the table because in the \code{Makefile} we include \code{./tables} directory in the \code{\$TEXINPUTS} environment variable.

\begin{table}
  \begin{center}
  \caption{Example table. \label{tab:example}}
  %\begin{ruledtabular}
  \begin{tabular}{lccc}
\hline\hline
Column 1 & Column 2 & Column 3 &  Column 4 \\[3pt]  
     &    $\deg$     & $\kpc$   &  $\deg$ \\[4pt]
\hline
Obj1 & (0,0) & 10 & 0.1 \\
... & ... & ... & ... \\
ObjN & (0,0) & 10 & 0.1
\\\hline\hline
\end{tabular}
\end{center}
%\end{ruledtabular}
\end{table}

%\begin{\tabletype}{l ccccccc }
%\tablewidth{0pt}
%\tabletypesize{\tiny}
%\tablecaption{ An example table. \label{tab:example}}
%\tablehead{
%(1) & (2) & (3) & (4) & (5) & (6) & (7) & (8)\\
%Name & GLON,GLAT & Distance & $r_{1/2}$ & $\log_{10}(J_{\rm meas})$ & $\log_{10}(J_{\rm pred})$ & Sample & Refrence \\
% & (deg) & (kpc) & (pc) & $\log_{10}(\GeV^2 \cm^{-5})$ & $\log_{10}(\GeV^2 \cm^{-5})$ & & 
%}
%\startdata
%Bootes I                     & 358.08,69.62   & 66  & 189  & $18.8 \pm 0.2$ & 18.5           & I,S,C & ... \\
%\\
%...\\
%\\
%Willman 1                    & 158.58,56.78   & 38  & 19   & $19.1 \pm 0.3$ & 18.9           & I,S & ... \\
%\enddata
%{\footnotesize \tablecomments{ (1) The first column. (2) The second column ...}}
%\end{\tabletype}


Equations appear as follows, and can be referred to as, for example, \eqnref{example} -- just as for tables, we use the \verb=\eqnref= command using the \code{eqn:} prefix.
\begin{equation}
  \label{eqn:example}
  \langle f(k) \rangle = \frac{ \sum_{t=0}^{N}f(t,k) }{N}
\end{equation}


% ----------------------------------------------------------------------

\section{Results}
\label{sec:results}

\figref{example} shows an example figure, referred to with the \verb=\figref= command and the \code{fig:} prefix.

\begin{figure}
\includegraphics[width=0.9\columnwidth]{example.png}
\caption{An example figure: the LSST DESC logo, copied from \code{texmf/logos/desc-logo.png} into \code{figures/example.png}. \label{fig:example}}
\end{figure}


% ----------------------------------------------------------------------

\section{Discussion}
\label{sec:discussion}

If you are planning on committing your paper to GitHub, it's a good idea to write your tex as one sentence per line.
This allows for an easier \code{diff} of changes.
It also makes sense to think of latex as \emph{code}, and sentences as logical statements, occupying one line each.
Each line must ``compile'' in the mind of the reader.


% ----------------------------------------------------------------------

\section{Conclusions}
\label{sec:conclusions}

Here's a summary of what we just reported.

We can draw the following well-organized and neatly-formatted conclusions:
\begin{itemize}
  \item This is important.
  \item We can measure some number with some precision.
  \item This has some implications.
\end{itemize}

Here are some parting thoughts.


% ----------------------------------------------------------------------

\subsection*{Acknowledgments}

Here is where you should add your specific acknowledgments, remembering that some standard thanks will be added via the \code{acknowledgments.tex} and \code{contributions.tex} files.

%
This paper has undergone internal review in the LSST Dark Energy Science Collaboration. We thank the reviewers: Mike Jarvis, Yao-Yuan Mao and Mariana Penna-Lima for comments that helped improved this manuscript and the \ccl library overall. We thank Matt Becker for helping us address the \ccl code review.

%DESC standard paper acknowledgements
%From https://github.com/LSSTDESC/desc-tex/blob/master/ack/standard.tex
The DESC acknowledges ongoing support from the Institut National de Physique Nucl\'eaire et de Physique des Particules in France; the Science \& Technology Facilities Council in the United Kingdom; and the Department of Energy, the National Science Foundation, and the LSST Corporation in the United States.  DESC uses resources of the IN2P3 Computing Center (CC-IN2P3--Lyon/Villeurbanne - France) funded by the Centre National de la Recherche Scientifique; the National Energy Research Scientific Computing Center, a DOE Office of Science User Facility supported by the Office of Science of the U.S.\ Department of Energy under Contract No.\ DE-AC02-05CH11231; STFC DiRAC HPC Facilities, funded by UK BIS National E-infrastructure capital grants; and the UK particle physics grid, supported by the GridPP Collaboration. This work was performed in part under DOE Contract DE-AC02-76SF00515. 

%In addition
We would like to thank the organisers of the DESC meetings and hack weeks in the period 2015-2018, where this work was partly developed. We would also like to acknowledge the contribution of the participants of the Theory and Joint Probes Code Comparison Project, some of whom are among the \ccl contributors, for providing the benchmarks for testing CCL. We also acknowledge Louis Penafiel and Elizabeth Kimura, who developed the {\tt VARRIC} code to compare and visualize power spectra calculated by \ccl and {\tt CLASS}. We thank Pedro Ferreira for feedback on this manuscript. We are grateful for the feedback received from other working groups of DESC, including Strong Lensing, Supernovae, Clusters and Photometric Redshifts. We are grateful to Katrin Heitmann and Earl Lawerence for discussions concerning the Cosmic Emulator. We are also thankful to the {\tt CLASS} authors and to Andrew Hamilton for making their codes available and allowing us to use them in this work. We thank Peter Williams for making his ApJ bibstyle file available\footnote{\url{https://github.com/pkgw/tex-stuff/blob/master/yahapj.bst}.}.  
%

DA is supported by the Science and Technology Facilities Council (STFC) through an Ernest Rutherford Fellowship, grant reference ST/P004474/1. NEC acknowledges support from a Beecroft fellowship and a Royal Astronomical Society Research Fellowship. TT acknowledges funding from the European Union's Horizon 2020 research and innovation programme under the Marie Sk{l}odowska-Curie grant agreement No.\ 797794. MI acknowledges that this material is based upon work supported in part by NSF under grant AST-1517768 and the U.S. Department of Energy, Office of Science, under Award Number DE-SC0019206.


Author contributions are listed below. \\
Husni Almoubayyed: Reviewed code/contributed to issues. \\
David Alonso: Co-led project; developed structure for angular power spectra; implemented autotools; integrated into LSS pipeline; contributed to: background, power spectrum, mass function, documentation and benchmarks; reviewed code \\
Jonathan Blazek: Planning capabilities and structure; documentation and testing. \\
Philip Bull: Implemented the Python wrapper and wrote documentation for it; general bug fixes, maintenance, and code review; enhanced the installer and error handling system. \\
Jean-\'Eric Campagne: Angpow builder and contributed to the interface with CCL. \\
N. Elisa Chisari: Co-led project, coordinated hack projects \& communication, contributed to: correlation function \& power spectrum implementation, documentation, and comparisons with benchmarks. \\
Alex Drlica-Wagner: Helped with document preparation. \\
Tim Eifler: Reviewed/tested code. \\
Ren\'ee Hlozek: Contributed initial code for error handling structures, reviewed other code edits. \\
Mustapha Ishak: Contributed to planning of code capabilities and structure; reviewed code; identified and fixed bugs. \\
David Kirkby: Writing, testing and reviewing code. Asking questions. \\
Elisabeth Krause: Initiated and co-led project; developed CLASS interface and error handling; contributed to other code; reviewed pull requests. \\
C. Danielle Leonard: Wrote and tested code for LSST specifications, user-defined photo-z interface, and support of neutrinos; reviewed other code; wrote text for this note. \\
Phil Marshall: Helped with document preparation. \\
Thomas McClintock: Wrote Python documentation. \\
Sean McLaughlin: Wrote doxygen documentation and fixed bugs/added functionality to distances. \\
J\'er\'emy Neveu: Contributed to Angpow and built the interface with CCL. \\
St\'ephane Plaszczynski: Contributed to Angpow and contributed to the interface with CCL. \\
Javier Sanchez: Modified setup.py to allow pip installation and uninstall. \\
Sukhdeep Singh: Contributed to the correlation functions code. \\
An\v{z}e Slosar: Wrote and reviewed code. \\
Antonio Villarreal: Contributed to initial benchmarking, halo mass function code, and general code and issues review. \\
Michal Vrastil: Wrote documentation and example code, reviewed code. \\
Joe Zuntz: Wrote initial infrastructure, C testing setup, and reviewed code. \\


%{\it Facilities:} \facility{LSST}

% Include both collaboration papers and external citations:
\bibliography{lsstdesc,main}

\end{document}
% ======================================================================
% 
